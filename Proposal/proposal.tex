\documentclass{article}
\usepackage[utf8]{inputenc}

\title{\textbf{Project Proposal:\\Optimal de-gerrymandering of voting districts}}
\author{\textsc{Rory Lipkis}}
\date{}
\newcommand{\uvec}[1]{\boldsymbol{\hat{\textbf{#1}}}}
\newcommand{\conj}[1]{{#1}^{\dagger}}
\newcommand{\del}{\nabla}
\newcommand{\D}{\mathrm{d}}
\newcommand{\TD}[2]{\frac{d#1}{d#2}}
\newcommand{\TTD}[2]{\frac{d^{2}{#1}}{d{#2}^{2}}}
\newcommand{\PD}[2]{\frac{\partial#1}{\partial#2}}
\newcommand{\PPD}[2]{\frac{\partial^{2}{#1}}{\partial{#2}^{2}}}
\newcommand{\PPPD}[2]{\frac{\partial^{3}{#1}}{\partial{#2}^{3}}}
\newcommand{\la}{\langle}
\newcommand{\ra}{\rangle}
\newcommand{\const}[1]{\Big\rvert_{#1}}
\newcommand{\pfrac}[2]{\left(\frac{#1}{#2}\right)}
\usepackage{fullpage}
\usepackage{amsmath}
\usepackage{amssymb}
\usepackage{setspace}
\usepackage{mathtools}
\usepackage{enumerate}
\usepackage{esint}
\usepackage{booktabs}
\usepackage{physics}
\usepackage{siunitx}
\usepackage{graphicx}
\usepackage{graphics}
\usepackage{dsfont}
\usepackage{slashed}

\begin{document}

\maketitle

\section*{Introduction}
Gerrymandering is the practice of modifying the boundaries of voting districts in order to produce an electoral outcome that does not reflect the actual distribution of preference in the population. Specifically, in systems in which each voting district elects a delegate to a congressional body (in winner-takes-all election), gerrymandering produces a gap between the preference distribution of the population and that of the congress.

This can be effected in a number of ways. A particular faction of voters can be divided in order to ensure that they remain a minority in any given district, and thus elect no delegates. Alternately, a faction can be consolidated in order to ensure that they hold a majority in only one district, and thus elect a single delegate. Gerrymandering does not necessarily entail convoluted boundaries. Different choices of seemingly unbiased boundaries, such as a simple grid, can produce wildly different outcomes, depending on the specific parameters and placement. This often leads to gridlock in discussions of gerrymandering, as there is rarely a configuration that is both simple and completely neutral.\footnote{Further complicating such discussions is the fact that one party typically has a vested interest in maintaining the disproportionate outcomes of gerrymandering, a phenomenon that this paper does not explore.}

\section*{Problem statement}
For a given population of $P$ people, each with a location $\mathbf{x}_{i} \in \mathbf{R}^{2}$ and affiliation $a_{i} \in \{1 \dots m\}$, draw $n$ voting districts such that:
\begin{enumerate}[(a)]
	\item Districts are convex.
	\item Districts have nearly equal populations.
	\item The set of majority affiliations has a composition maximally similar to that of the general population.
\end{enumerate}

\section*{Discussion and proposed method}
A sensible approach to the problem is to use Voronoi regions, which automatically satisfy the convexity requirement. The search space is then reduced to $\mathbf{R}^{2n}$, as the optimization occurs over the coordinates of the $n$ Voronoi centers. The equal population requirement can either be imposed as a constraint that population of each district lie within $\delta$ of $P/n$, or as a penalty in the cost function. For the optimization scheme, some sort of direct method will be employed, likely a population method. This will require significant experimentation. 

Initially, the optimization will be performed with $m = 2$, which most closely mirrors the American reality. If time permits, $m > 2$ cases will be explored as well. Optimization success will be determined by the achieved proportional similarity of population and congressional compositions. A significant portion of the paper will also be devoted to characterizing the stability of the solutions. If slight local changes to the population result in a global redrawing of districts, the solution cannot be considered robust.

\end{document}